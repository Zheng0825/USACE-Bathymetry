Historically, the standard approach for resolving coastal hydrodynamics has been to directly measure bathymetry and use the data as input in numerical models to determine coastal conditions. However, direct measurements of bathymetry are known to be costly in both a temporal and financial sense; requiring heavy project hours and sophisticated equipment. In recent years, remote sensing and photogrammetry have seen rapid advancement, facillitating the easy measurement of surface properties of the coastal environment.

Using three different inversion algorithms, we approximate bathymetry using linear wave theory.

 We are using a mathematical approach to estimate bathymetry because waves make it difficult to determine the depth of the ocean physically with technology. Our three methods, nonlinear least squares, Tikhonov regularization, and Bayesian MCMC, prove to have promising results for the estimation of depth given real wave number data. 


[Need to solve the following problem with priors $\mathbf{h}_p$].

\begin{equation}\label{LS-regBC}
\mathbf{\hat{h}} = \underset{\mathbf{0} \preceq \mathbf{h} \preceq \mathbf{11}}{\arg \min} \ \ \|  \mathbf{A}(\mathbf{h}) -  \mathbf{d} \|_2^2  +  \alpha \| \mathbf{h} -  \mathbf{h}_p\|_2^2,
\end{equation}

[Want to use proper method to find the optimum regularization parameter $\alpha$]

\begin{itemize}
\item Briefly summarize your contributions, and their possible
impact on the field (but don't just repeat the abstract or introduction).
\item Identify the limitations of your approach.
\item Suggest improvements for future work.
\item Outline open problems.
\end{itemize}


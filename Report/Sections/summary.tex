Historically, the standard approach for resolving coastal hydrodynamics has been to directly measure bathymetry and use the data as input in numerical models to determine coastal conditions. However, direct measurements of bathymetry are known to be costly in both a temporal and financial sense requiring heavy project hours and/or sophisticated equipment. In recent years, remote sensing and photogrammetry have seen rapid advancement, facillitating the easy measurement of surface properties of the ocean. This availability of data has motivated the scientific community to try to flip problem and estimate bathymetry using surface properties.

For this project we propose three different inversion methods for estimating bathymetry from wavenumber data; a standard Non-linear Least Squares approach, a Bayesian MCMC statistical method, and a Tikhonov Regularized Non-Linear Least Squares method. All three methods perform well in the dynamic portion of the near shore (the region of highest interest) and accurately predict the location of the sand bar at approximately $x = 900$ m. As we move further off shore we see notable deviations from the known bathymetry. This deviation has been attributed to the fact that as depth increases, wave numbers become less dependent on water depth due to limited interaction of surface waves with the sea bed. This phenomenon leads to an ill-conditioning of our forward operator and thus small deviations in the measured wavenumber lead to large fluctuations predicted bathymetry below. Of the three methods tested, the most promising to deal with this difficulty is the Tikhonov Regularization scheme as the use of a regularizer is well equipped to handle the instabilities introduced by this ill-conditioning, as shown in both the synthetic and real data tests. That being said, all three methods perform well in the region of highest interest and further work should be undertaken to test to suitability of each.

To expand on this work, there are many options for future work. In Section (FORWARD PB) we assume the dispersion relationship given by linear wave theory but formulations that involve wave height, such as those in (CITE NON-LIN), should be tested to assess the validity of the linear model in this context. Additionally, the inversion methods outlined in this report will allow for the use of other measured parameters in the assimilation scheme, such as wave height or known bathymetry. The use of multiple measurement types may reduce the sensitivity to noise in a single variable. 


\begin{itemize}
\item Briefly summarize your contributions, and their possible
impact on the field (but don't just repeat the abstract or introduction).
\item Identify the limitations of your approach.
\item Suggest improvements for future work.
\item Outline open problems.
\end{itemize}


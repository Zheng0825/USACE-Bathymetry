Bathymetry is a measurement of submarine topography and can be used to understand shifts of the ocean floor and its depth. Knowledge of bathymetry is important for marine navigation, both civilian and military, as well as for monitoring the effects of storms on coastal environments. While direct measurement of bathymetry is possible, the process tends to be cost and time prohibitive. For example, amphibious vehicles (see Figure~\ref{crablarc}) are capable of spatially limited surveys of bathymetry in difficult surf-zone conditions but require significant resources to operate. Resulting surveys tend to be sparse in time as well due to these considerations. 

More desirable would be a method to estimate bathymetry using surface measurements collected via remote sensing such as airborne and satellite platforms. While bathymetry data is currently sparse due to observational limitations, the physics of waves are reasonably well understood. In particular, a dispersion relationship can be used to relate water depth to surface properties such as wave length and wave period. It is therefore possible to estimate bathymetry given observations of the water surface. Light Detection And Ranging (LIDAR) has been used to determine wave heights and Argus land-mounted video has been analyzed photogrammetrically to determine wave frequency and wave number. Both of these sources therefore provide valuable inputs for estimating coastal bathymetry in a more efficient manner than is currently available.

%GPS-equipped jet skis with fathometers use echo sounding to determine water depth,

Both wave and bathymetric data has been collected in Duck, NC by the U.S. Army Corps of Engineers Coastal and Hydraulics Laboratory, including in situ measurements of bathymetry and measurements of the water surface. This is a useful case for testing algorithms to invert for bathymetry because the true bathymetry is available for comparison to numerical estimates.

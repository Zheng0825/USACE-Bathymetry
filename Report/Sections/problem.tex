%==============================================

Although there have been uncertainties in capturing the topography of the ocean nearshore, mathematical methods could prove to be possible solutions to this problem using the dispersion relationship between wavelength and the period. Stockdon and Holman used video imagery, which compared true wave signal and remotely sensed video signal to create a linear representation between wave amplitudes and phases. Holman used a 2-dimenisional method with Kalman filtering to estimate the depth, $h$.

Our research will compute the wave depth using wave length and wave number with a 1D model derived from using the energy flux method to create a correlation between the wave length and the wave depth from the surface.
\subsection{Forward Problem}\label{forwardproblem}
We consider the following models for the forward problem.
\begin{eqnarray}
\label{fp1}
\left \{
\begin{array}{lll}
\frac{d}{dx}\left(EC_g\right)=-\delta,\\
\\
\sigma^2=gk\tanh(kh),
\label{ode}
\end{array}
\right.
\end{eqnarray}
where the speed at which the energy is transmitted, $C_g$, called linear theory group speed, is given as
\begin{equation}
\label{cg}
C_g=\frac{c}{2}\left(1+\frac{2kh}{\sinh(2kh)}\right),
\end{equation}
with $c=\frac{\sigma}{k}$ and from the linear theory, the wave energy, $E$ is given as
\begin{equation}
\label{e}
E=\frac{1}{8}\rho g H^2,
\end{equation}
$\rho$ is the density of water, $g$ is the gravitational acceleration, $\sigma$ is the angular frequency, $c$ is the local wave phase speed, $k$ is the wave number, $h$ is the water depth and $H$ is the wave height. Here, we assume $T$, the wave period, is constant.\\ 

\noindent Observe that equation \ref{fp1} is coupled using the fact that $\sigma=\frac{2\pi}{T}$. Hence, using equation \ref{cg} and \ref{e} in \ref{fp1}, we obtain
\begin{equation}
\label{fpdelta}
\frac{d}{dx}\left( \frac{\lambda}{k}\left(1+\frac{2kh}{\sinh(2kh)}\right)H^2 \right)=-\delta,
\end{equation}  
and 
\begin{equation}
\label{fk}
f(k) = gk\tanh(kh)-\sigma^2,
\end{equation}
where $k$ is the zero of function $f$ and $\lambda=\frac{\rho g \pi}{8T}$.\\
The wave breaking, $\delta$, proposed by (Janssen and Battjes, 2007) given as
\begin{equation}
\delta = \frac{1}{4h}B\rho g f H_{rms}\left[(R^3+\frac{3}{2}R)e^{-R^2}+\frac{3}{4}\sqrt{\pi}(1-erf(R))\right],
\end{equation}
where $$R=\frac{H_b}{H_{rms}}, \quad H_b = 2\sqrt{2}H_{m0},\quad H_b=0.78h,\quad f=\frac{1}{T},\quad B=1.$$
\subsection{Numerical Solution of the Forward Model}
Finite difference scheme is applied to obtain numerical solution with appropriate initial and boundary conditions. In the scheme of finite differences, the derivatives are replaced using their finite-difference approximations. The goal is to provide wave height using the finite difference method. In the process, MATLAB fsolve function is applied to obtain the wave number. Furthermore, Newton-Raphson method is used to verify solution of wave number $k$ obtained by the fsolve function.
\subsubsection{Discretization of the Model}
Forward difference is applied depending on the nature of the model given in equation (\ref{fpdelta}).\\
Let 
$$c^{*} = \frac{c_{g}\lambda}{k},\quad c_{g} = nc =\frac{2\pi n}{T},$$
with 
$$\lambda = \frac{1}{8}\rho g, \quad n = \frac{1}{2}\left(1+\frac{2hk}{sinh(2hk)}\right).$$
Let $\widetilde{H}=H^{2}$. Then the ordinary differential equation of the model given in (\ref{fpdelta}) becomes
\begin{equation}
\frac{d}{dx}\left( c^{*}\widetilde{H}\right)=-\delta
\end{equation}
Applying the finite difference method, the above expression becomes
$$
\frac{c_{i}^{*}\widetilde{H}_{i}-c_{i-1}^{*}\tilde{H}_{i-1}}{\triangle x}= \delta \quad \Rightarrow \quad
\widetilde{H}_{i}=\frac{c_{i-1}^{*}\widetilde{H}_{i-1}+\triangle x \delta}{c_{i}^{*}}
$$
Hence, at each index point in the discretization, $
H=\sqrt{\widetilde{H}}$.
\subsubsection*{MATLAB function fsolve}
As part of the process of obtaining wave height, $(H)$, MATLAB function fsolve is used as non-linear solver to find the zeros of the function given in (\ref{fk}) obtained from the dispersion relationship
$$
\sigma^{2}=gk tanh(kh).
$$ 
So, at each index points, wave number $(k)$ is generated with initial guess as
$$k_0=\frac{\sigma}{\sqrt{gh}}.$$
\subsubsection*{Newton-Raphson Method}
Newton-Raphson method is a widely used method for roots finding. This method is used in the numerical experiment to verify the wave number extracted using MATLAB function fsolve. Therefore, approximate solution using Newton-Raphson method is obtained as
\begin{eqnarray}
k_{i+1}& =& k_{i}-\frac{gk_i\tanh(k_ih)-\sigma^2}{g\tanh(k_ih)-ghk_i sech^2(k_ih)},
\end{eqnarray}
using the same initial guess as for fsolve.
This provides the wave number, $(k)$, in each index.

\subsubsection{Implementation}
To apply the finite difference method we first discretize the space vector ${x}$ depending on predefined mesh size ${\Delta x}$. For test purposes ${\Delta x}$ is considered as ${10m}$ which means this will estimate the wave height in every 10 meter. The finite difference scheme provides a sparse bidiagonal matrix. In this experiment wave break condition is applied as a cap value to estimate the wave height at index points. The wave break condition applied in this experiment as
$${H=0.78h}$$
%The reason for this condition is the model is focused on the shallow water. In shallow water the individual wave breaks when the wave height  ${(H)}$ and depth ${h}$ relationship is 
%$${H>0.8h}$$

The implementation of the algorithm is as follows
\begin{algorithm}
\caption{Algorithm to estimate wave height H}\label{euclid}
\begin{algorithmic}[1]
\Procedure{}{}
\BState \emph{\textit{\textbf{Initialization}}}:

\State $\textit{Initial depth:\,\,} h$
\State $\textit{Wave period:\,\,} T$
\State $\textit{Wave number estimated from fsolve:\,\,}k$
\State $\textit{Wave breaking:\,\,}\delta$

\BState \emph{\textbf{Step 1:}}
\State $\bullet \quad\textit{constant}\quad \lambda=\frac{ \rho g \pi}{8T}$

\BState \emph{\textbf{Step 2:}}
\State $\bullet \quad\textit{Find\,\,} c^{*} \textit{\,\,depending on \,\,}k:\quad c^{*}=\frac{(1+(2kh)}{sinh(2kh)}\lambda \,\, k$

\BState \emph{\textbf{Step 3:Compute wave height H}}
\State $\bullet \quad\textit{Compute:\,\,}\widetilde{H}:\quad \widetilde{H}_{i}=\frac{c_{i-1}^{*}\widetilde{H}_{i-1}+\triangle x \delta}{c_{i}^{*}}$
\State $\bullet \quad\textit{Compute wave break condition\,\,}Hmax:0.78 h$

\State $\bullet \quad\textit{Take minimum between\,}\widetilde{H} \textit{\, and \,\,} Hmax \textit{\,\,at each index\,\,}i\quad $

\EndProcedure
\end{algorithmic}
\end{algorithm} 


\subsubsection{Numerical Results}

Although there have been uncertainties in capturing the topography of the ocean nearshore, mathematical methods could prove to be possible solutions to this problem using the dispersion relationship between wavelength and the period. Stockdon and Holman used video imagery, which compared true wave signal and remotely sensed video signal to create a linear representation between wave amplitudes and phases. Holman used a 2-dimenisional method with Kalman filtering to estimate the depth, $h$.

Our research will compute the wave depth using wave length and wave number with a 1D model derived from using the energy flux method to create a correlation between the wave length and the wave depth from the surface.
\subsection{Forward Problem}
We consider the following models for the forward problem.
\begin{eqnarray}
\label{fp1}
\left \{
\begin{array}{lll}
\frac{d}{dx}\left(EC_g\right)=\delta,\\
\\
\sigma^2=gk\tanh(kh),
\label{ode}
\end{array}
\right.
\end{eqnarray}
where the speed at which the energy is transmitted, $C_g$, called linear theory group speed is given as
\begin{equation}
\label{cg}
C_g=\frac{c}{2}\left(1+\frac{2kh}{\sinh(2kh)}\right),
\end{equation}
with $c=\frac{\sigma}{k}$ and from the linear theory, the wave energy, $E$ is given as
\begin{equation}
\label{e}
E=\frac{1}{8}\rho g H^2,
\end{equation}
$\rho$ is the density of water, $g$ is the gravitational acceleration, $\sigma$ is the angular frequency, $c$ is the local wave phase speed, $k$ is the wave number, $h$ is the water depth and $H$ is the wave height. Here, we assume $T$, the wave period, is constant.\\ 


\noindent Observe that equation \ref{fp1} is coupled using the fact that $\sigma=\frac{2\pi}{T}$. Hence, using equation \ref{cg} and \ref{e} in \ref{fp1}, we obtain
\begin{equation}
\frac{d}{dx}\left( \frac{\lambda}{k}\left(1+\frac{2kh}{\sinh(2kh)}\right)H^2 \right)=-\delta,
\end{equation}  
and 
\begin{equation}
f(k) = gk\tanh(kh)-\sigma^2,
\end{equation}
where $k$ is the zero of function $f$ and $\lambda=\frac{\rho g \pi}{8T}$.\\
The wave breaking, $\delta$, proposed by (Janssen and Battjes, 2007) given as
\begin{equation}
\delta = \frac{1}{4h}B\rho g f H_{rms}\left[(R^3+\frac{3}{2}R)e^{-R^2}+\frac{3}{4}\sqrt{\pi}(1-erf(R))\right],
\end{equation}
where $$R=\frac{H_b}{H_{rms}}, \quad H_b = 2\sqrt{2}H_{m0},\quad H_b=0.78h,\quad f=\frac{1}{T},\quad B=1.$$
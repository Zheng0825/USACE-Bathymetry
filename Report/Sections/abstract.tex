Tracking coastal bathymetry is necessary for marine navigation, military activities, and assessment and prediction of storm damage as beaches evolve. Past efforts have derived surface wave properties from in situ bathymetric measurements. However, in situ measurements are costly and laborious to collect. As a result, direct observations of bathymetry are sparse in time and space. On the other hand, remotely sensed observations of surface conditions are becoming easier to obtain. We seek methods to invert for bathymetry given surface conditions. 

We utilize linearized wave theory to estimate bathymetry near Duck, North Carolina given measurements of surface wave properties collected by the U.S. Army Corps of Engineers. We process observations of wave height, wave number, and bathymetry for assimilation into and validation of computational models of wave mechanics. We create a forward model to estimate wave number and wave height given bathymetry information. Several inverse methods including nonlinear least squares, Bayesian Markov Chain Monte Carlo, and Tikhonov regularization then use the surface wave data and forward model to estimate bathymetry along a one-dimensional profile.

Our results demonstrate reasonable estimates of depth, $h$, in the near shore region, within 500 m of the beach. All inverse methods are able to accurately reconstruct a sandbar located in this region which is an important feature for consideration of rip tides and coastal navigation. Accuracy of the methods drops off past $\sim$550 m from shore due to the increased depth and subsequent lower sensitivity of wave number to depth. We suggest several avenues for possible expansion of this work in the future which may improve overall accuracy of bathymetry estimates.